\begin{abstract} 
This paper studies how organizational responses to policies promoting women were shaped by wartime exposure to female coworkers in the Tokyo civil service. We assemble a new personnel panel covering more than 330,000 public servants over 35 years and link wartime assignments to postwar workplace composition and careers. Identification exploits office-level shocks from military conscription: when men were drafted, managers reshuffled staff and hired replacements, creating plausibly exogenous variation in contact with women. We find that offices led by managers with greater wartime exposure were more likely to hire and retain women after restrictions were lifted, and that exposure increased the likelihood that veterans later worked with women. These effects persist, suggesting that temporary disruptions can change workplace norms and accelerate organizational diversification.
\end{abstract}