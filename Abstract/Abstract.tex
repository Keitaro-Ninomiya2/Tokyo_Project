\begin{abstract} 
This paper examines the long-term impact of increased female representation on organizational gender dynamics, using the unique historical context of World War II in Tokyo. Leveraging a newly constructed personnel panel dataset of over 330,000 public servants spanning 35 years, we explore how the conscription of male workers and the subsequent entry of women into the workforce reshaped gender composition within organizations. Employing variations in office-level disruptions caused by military drafting as an instrumental variable, we provide causal estimates of how exposure to female colleagues influences employees’ future career trajectories and office gender diversity. Our findings demonstrate that increased exposure to women during the war persistently improved gender balance in offices, suggestively driven by shifts in hiring practices and co-working experiences. These results highlight the role of temporary shocks in accelerating workplace diversification and underscore the potential for external disruptions to promote long-term organizational change.
\end{abstract}