\subsection{Institutional Details of Tokyo Civil Service}

The Tokyo public service, comprising the \emph{Tokyo-Fu} and \emph{Tokyo-Shi}, managed public good provision across geographically distinct areas of Tokyo. Although these bodies operated over separate regions, their main offices were centrally located in downtown Tokyo (Figure 1).

In 1944, under intensifying war pressures and facing repeated military defeats, the central government mandated the amalgamation of \emph{Tokyo-Fu} and \emph{Tokyo-Shi} into a single entity, named \emph{Tokyo-To}, through a federal bill. This merger, driven by operational inefficiencies and resource allocation challenges during the war, resulted in significant layoffs, particularly affecting \emph{Tokyo-Shi} employees.

\emph{Tokyo-To} employed approximately 15,000 highly educated staff, with 30 percent hired through national exams and the remainder locally. The organizational structure was hierarchical, dividing departments based on the public services provided, with some departments split into as many as four layers. Employees were assigned an occupation, such as engineers, managers, clerks, or assistants, along with a classification reflecting their professional expertise and experience. These assignments were determined upon entry into the organization, although transitions across occupations occasionally occurred. The recruitment process for non-assistant employees was merit-based, relying on competitive examinations and internal assessments. Within the civil service, workers were further categorized into upper and lower classes, with managers representing the upper class and clerks or engineers forming the lower class. Leadership positions, such as department heads, were typically filled by personnel recruited from the central government. For many employees, the typical career path involved progressing step by step up the job ladder, with clerks serving as the standard entry point for civil servants aiming to advance. However, the level of competition for promotions among employees was likely influenced by the conscription of a significant portion of the workforce into military service, which temporarily reduced the pool of available candidates. This shift may have also led to substantial changes in both the promotion process and the speed of advancement for positions such as clerks and managerial staff. In fact, many staff members were conscripted as soldiers, resulting in a severe manpower shortage for managing wartime administration (Tokyo Metropolitan Government 2013). 

\subsection{Drafting Process of the Japanese Military Service}
The Japanese military assembled its forces by calling active service troops and reservists, a process pivotal to our study’s analysis of workforce dynamics during and post-conflict periods. The conscription system was designed to ensure a steady supply of personnel by mandating physical examinations for all eligible men upon reaching the age of 20, with those meeting the criteria randomly selected for service.

The drafting process, as outlined in Figure 3, was structured into several stages to efficiently classify conscripts based on their physical capabilities and the military's operational needs. Initially, all eligible men underwent a physical examination to determine their fitness for active duty. Those deemed fit were assigned to active service, where they served as soldiers for a fixed period. Upon completing active duty, these individuals transitioned to the reservist category, ensuring they remained available for military recall during national emergencies. For individuals who did not meet the physical standards for active duty, an alternative route, termed replacement service, was established. These men received training but were not immediately deployed, instead forming a reserve pool for supplementary support when required. Additionally, a subset of candidates was fully exempted from military obligations due to severe health conditions or other qualifying factors.

As the war intensified, particularly in the wake of Japan’s challenges during the Second Sino-Japanese War (1937–1945), the drafting system underwent significant modifications to address the military's increasing personnel demands. In 1943, reserve obligations were extended: the period of reserve service following active duty increased from 5–6 years to 15 years, and the secondary reserve period was prolonged until the age of 45. These changes ensured that the military could rely on a broader pool of trained personnel over an extended timeframe, providing critical support for wartime operations.

\subsection{The Impact of the War on Tokyo-To}
The war significantly altered the workforce composition within \emph{Tokyo-To}, particularly affecting female participation rates and the distribution of staff across various offices. The conscription system, which required eligible men to serve in the military, had profound implications for public institutions, especially in urban centers like Tokyo. Male workers in administrative, engineering, and clerical roles were not exempt from military obligations, leading to their systematic removal from public service positions as they were drafted into active or reserve duty. This mass conscription created substantial disruptions in the operations of public offices and left critical gaps in the workforce that urgently needed to be addressed.

During the war period, the entry of female workers also became a notable feature of \emph{Tokyo-To's} workforce. Before the war, female employees were largely concentrated in specific departments and were almost absent from the core administrative divisions of the metropolitan government. Their roles were typically limited to auxiliary or supporting tasks, reflecting the gendered labor practices of the time. However, the rapid expansion of labor demand driven by conscription accelerated the entry of women into the workforce. Units were generally structured to include a manager, clerks or engineers, and assistants. Female workers were primarily hired as assistants to support unit leaders in their tasks, often in offices with a lower concentration of engineers. Their continuation and promotion within the organization depended heavily on the approval of their respective unit managers, with their relationships with male incumbents—such as direct subordinates to unit managers or co-workers to clerks and engineers—playing a significant role. In addition to conscription, the demands of wartime administration likely necessitated an increase in the number of staff employed by \emph{Tokyo-To}. The rising responsibilities of the metropolitan government during the war made it essential to expand its workforce. Historical evidence suggest that both the city and the prefecture increased the number of employees and departments during this period (Tokyo Metropolitan Government 2013).

To compensate for the depletion of male workers, \emph{Tokyo-To} rapidly increased the recruitment of women, who were not subject to military service. Women entered the public workforce in unprecedented numbers, particularly in assistant roles, as highlighted in Figure 5. Panel (a) illustrates the sharp increase in the population of female public civil servants during the World War II (WWII) period, marked by the shaded red area. The count of female employees rose significantly between 1938 and 1944, reflecting a direct response to the conscription of men. However, this trend reversed immediately after the war as male workers returned to their civilian roles, leading to a sharp decline in the number of female public civil servants.

Panel (b) of Figure 5 provides further insight by examining the share of women in the public sector workforce during the same period. The proportion of female public civil servants also increased markedly during WWII, albeit remaining relatively small compared to the overall workforce. The rise in this share highlights not only the increasing reliance on women but also the gendered constraints of wartime employment policies, which often relegated women to lower-ranking and temporary roles within the organizational hierarchy. This unequal distribution persisted even after the war, as men resumed their pre-war positions, and women were often pushed out of public service.

Table 1 shows the difference between offices in 1936 with and without employees that were drafted 2 years later. We find a statistically significant different between the share of female workers between offices categories. Workers in future drafted offices were more likely to be occupied with male. Since the Japanese military service only conscripted males, the difference in gender balance between offices validates the drafting pattern. The two office categories matches each other with respect to its share of new workers and share of higher-ranked workers. We fail to reject similarities between the at least at the 15th confidence level.  Drafted and non-drafted employees worked in offices with nearly half of their peers hired at most 4 years before. Also they did not differ in the share of higher-class public servants that they reported to.

