\begin{table}[ht]
\centering
\begin{tabular}{lrrrrrl}
  \hline
 Year 1936 & Not Drafted & Drafted & Difference & t statistic & p value &  \\ 
  \hline
  Female workers share & 0.04 & 0.02 & -0.03 & 3.07 & 0.00 & ** \\ 
  New hire share & 0.49 & 0.53 & 0.04 & -1.09 & 0.28 &  \\ 
  Share of higher class employees & 0.06 & 0.09 & 0.03 & -1.41 & 0.16 &  \\ 
  Total workers & 18.02 & 56.48 & 38.46 & -4.96 & 0.00 & *** \\ 
   \hline 
\multicolumn{6}{p{0.8\textwidth}}{\footnotesize{\textit{Note}: The two tables test the differences in office characteristics of employees that was drafted during the war period (1937-1944). Samples are conditional on employees working in the Tokyo Civil service during the war period. New hire share is the share of employees not appearing in previous year records. Higher class employees are civil servants in positions reserved for employees entering through the national exam. Total worker is defined at the office level. p-values test the null hypothesis that the difference is zero.   *** p $<$ 0.001, ** p $<$ 0.01, * p $<$ 0.05.}} 
\end{tabular}
\end{table}
