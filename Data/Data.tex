Our primary data set comes from a historical archive of directory of staff of the Tokyo public servants. The archives include information about the name of the employee, their positions, and their home address. We infer the gender of the staff by using an algorithm trained on historical Japanese names. The personnel record covers from 1928 to 1958. 

The directory of staff was published annually with a full list of their employees. The two co-governing institutions (\emph{Tokyo-Shi} and \emph{Tokyo-Fu}) both published separately until the merger of the two institutions in 1943. The Tokyo-Fu directory contains a list of the employees that was drafted into the war. Since we observe the positions of all the employees in a institution, we identify the drafted individual and their surrounding co-workers in a directory. 

Our secondary data is the biographical directory of staff at 1942. The biography lists detailed information about higher ranked employees. It includes (1) Date of birth, (2) Prefecture of origin, (3) Education attainment (name of graduating institution) (4) Starting date of entering the public servant system, (5) hobbies, and (6) Physical address. However, the directory was not published annually like the directories, but only published in some years. Also only the Tokyo-Shi and Tokyo-To published biographical directories of their employees. In addition the biographical directory only reports the details for employees at least in positions higher than a clerk. 

We develop an algorithm that integrates multiple neural networks and computer vision technologies to detect complex layout structures in archival records and extract categorized text data. Given that most modern optical character recognition (OCR) systems are primarily trained on the English language, we utilize a neural network-based OCR developed by the National Diet Library in Japan. This OCR is adept at identifying texts within a document page, which we further analyze using named entity recognition technology to categorize information based on content and layout. We apply this algorithm to a series of historical documents to augment the biographical details of employees. This process generates a comprehensive panel data set of public sector employees in historic Tokyo, detailing their internal assignments by occupation and rank, along with biographic details for a subset of these employees.

Our OCR works better for the recent years than the older years. Since the calligraphy of Japanese letters changed after the war, a modern OCR fails to accurately identify the letters in the directory. Eventhough we use the NDL OCR, which is trained on historical Japanese letters, due to the resolution of the scanned images, the OCR sometimes fail to identify the content. However, we di recover approximately 70\% of employees in the historic era, and almost everyone for those in the more modern era (Figure 7).

We predict the gender of the employee by using an external named entity recognition algorithm that predicts the gender of the name. In contrast to Western names, Japanese names tend to reveal the gender of the employee easily. We measure the size and share of female workers of each office unit by aggregating over the pool of employees belonging to a given group. We measure the length of tenure for each employee that we identify in the directory by searching for the first and last year of the directory that we identify their name. Consequently, if an employee is missing in a directory between two directories in which the employee is detected, we treat the employee to be hired in the missing directory. We also infer the location of the office that an employee commuted to by referring to the name of the office unit. 

The digitizization of the historical archival documents leaves us with a sample of 333,000 observations across 35 years. The data identifies 126,372 unique names and 15,406 female names.