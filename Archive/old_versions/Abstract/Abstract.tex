\begin{abstract} This paper explores the impact of increased female representation on organizational gender inequality. We create a comprehensive personnel panel dataset (n $\geq$ 330,000) of public servants in Tokyo spanning 35 years and trace the augmentation of women's presence during the war and its enduring effects. We provide causal estimates of how exposure to female colleagues affects employees' future career trajectories, leveraging variations in office composition prompted by military drafting. Our analysis reveals that an increased proportion of women in an office enhances the gender balance within the future co-worker pool of employees. These findings underscore how wartime conditions compel organizations to reallocate internal labor resources and establish pathways for integrating underrepresented groups into their workforce. \end{abstract}