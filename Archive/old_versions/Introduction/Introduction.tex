The modern state of gender inequality is influenced by historical shocks, particularly large-scale conflicts that have reshaped labor markets. For example, research indicates that regions with higher male casualties during World War II experienced an increase in female labor participation (Acemoglu, Autor, and Lyle, 2003; Goldin and Olivetti, 2013). However, the specific mechanisms by which institutions integrated women into their workforces remain largely undocumented due to the absence of detailed, historical personnel data. A comprehensive, longitudinal dataset tracking individual employment histories would enable researchers to examine the distribution of female workers within institutions and trace their diffusion across offices and occupations, thereby clarifying the pathways linking these historical events to the current gender inequality landscape.

We construct a novel comprehensive employee level data with details on position, occupation, and gender, of public servants in historic Tokyo to identify the positional distribution of female workers within an institution and trace their expansion over decades of years. We digitize a series of archival documents using recent advances in neural network-based optical character recognition (OCR), named entity recognition, and computer vision technology (Dell, 2024). Our data consists of 126,000 employees across more than 500 offices throughout 35 years covering prewar, duringwar, and postwar periods. We identified approximately 15,000 women among the samples, along with the details of their male coworkers that shared office. We also identify the conscription status of each employee that allows us to identify the offices with any drafted workers. We observe 870 draftees with diverse backgrounds, such as their prewar workplace experience and occupation. The data set enables us to track the diffusion of female workers across offices for multiple decades, allowing us to identify notable feature of the diffusion of women in the workplace, and connecting the contrasts in gender balance across employees' career trajectories with their war-time experiences and their backgrounds.


Using this dataset, we study how conscription reshaped employee demographics through new hires and whether the wartime entry of women altered male coworkers' career trajectories, potentially shifting workplace gender balance. Similarly to other countries during WW\upperRoman{2},  the Japanese labor market accommodated female workers as a major part of their labor force due to the war-induced labor shortage. The Tokyo civil service followed a similar pattern as the war triggered the rapid entry of female workers into their workplace, potentially caused by the increase of labor demand due to the war and military duties pulled incumbent workers out of their workplace (Figure 1). Incidentally, since males historically dominated most of Tokyo's public service positions, the war led incumbent males to work with women for the first time in their career. The co-working experience potentially changed the gender balance of their future co-workers by accepting roles in the service with more gender balanced or hiring women into vacant position that was traditionally held by men, with lasting impacts on the employees' future career trajectory. The granular positional information of the entire workforce in Tokyo allows us to document the immediate reaction of the Tokyo public service to the severe shortage of labor and also allows us to trace the careers of the impacted workers and quantify their organizational impacts.




%Using this dataset, we investigate whether a co-working experience with women changes the career trajectories and the composition of their future co-worker pool. An increase in the presence of women prompts incumbent males to revise their perceptions on working with women while also reducing barriers for future female workforce entry. We test whether positively revising an employee's perspective towards working with women shifts their career towards a more gender diverse one. Employees with more accommodating composure towards women may object less to hiring a women for filling a vacant position and try to facilitate a new female hire to settle and conduct her work. Employees with altruistic behavior towards accommodating women into their workplace may help themselves promote further up by improving their offices performance. The rich personnel record of employees allow us to investigate how different work experience with peers and their demographics affect the future trajectories of their career.

In the first part of this paper, we test the short-term effect of losing an employee through drafting on the number of female workers in the affected office. The conscription of Japanese military personnel was roughly random, because the government did not reveal the drafting process, and therefore eligible men had little knowledge of their conscription chances. Our data show which offices had a drafted employee, which we use to identify the offices with any of their employees removed. Since the drafting process is largely random, the difference in the hiring pattern is driven by the drafting and not by other unobserved factors. Using a Poisson regression, we regress the number of female workers among the new hires against the number of incumbent workers drafted during the payroll year. To avoid confounding the hiring pattern with other unobservable office or worker characteristics, we include division fixed effects and occupation fixed effects to compare across offices with similar characteristics. We also provide balanced tables for the attributes of drafted and non-drafted workers to show that the two employee bodies are similar to each other. The immediate responses of the workforce against their employees' conscription reveal whether the entry of women into the workplace was instrumental in diversifying their workplace. 

We find that conscription increased the number of new hired women in the workplace. An increase of workplace disruption of a single employee exiting his position caused the number of new female employees hired in the same office by 15\%. The unit increase causes 13\% an increase in the number of new hires. By combining the results, I find that the conscription had minimal impact on the gender of the new hire. A unit increase in the number of drafted workers increased the share of women among new hires by merely 0.6 percentage points. The results indicates that drafting did expose men to more women by opening positions in their workplace that could potentially be filled by women, however the drafting did not change the share of female workers among the new hires. 

In the second part of the paper, we test the long-term of effect of conscription on the gender balance of the pool of employees future co-workers. We connect the employees during the war period with drafted co-workers and observe how the composition of their offices differs from those without any drafted co-workers. We use a Poisson model to estimate the impact of losing a coworker through drafting on the number of female workers in the workers future position. We also test whether short-term exposure to women during the war directly leads to the impacted employee having a different gender composition in their future offices. We use an instrumental variable approach to identify the causal effect of working with more women on the number of women in their future offices. The significant short reaction to drafting during the war period allows us to test its impact on gender balance of employees future co-worker, and additionally test whether the co-working with women shifted the employees career towards a career with more women.

We provide evidence of a long-term effect on the number of female workers of an impacted worker. A unit difference in the number of drafted workers during the war led to a positive difference in comparison to the number of female workers in their future position. However, the difference fades away approximately 15 years after the war. We do not find strongly positive effects on the share of female employees, indicating that the drafting did not lead to diverging paths of co-worker composition across employees. Our IV estimates reveal that the increase was not driven by greater exposure to women. Despite the positive estimate for the OLS, which is potentially biased due to selection in offices, the IV does not support the hypothesis that greater exposure to women directly leads to those male employees shifting into career trajectories with more women. 

Our results imply that the disruptions in workplaces caused by war accelerated the churn of employees during the war and caused different career trajectories across employees. Our first results imply that impacted office immediately seeking a new employee to fill the positions. Due to the shortage of male labor in the labor market, female worker's were occasionally hired to fill in the vacancies, and unintentionally exposed incumbent workers with more women in their offices. However, the male workers were more likely to fill in vacant positions than men, implying that traditional customs of hiring men for public service positions still remained strong during the war. Our second results implies that the disruption caused by conscription lead to co-workers of drafted employees to work in more gender balanced offices in their future. However, the results are not explained by the number of female workers that they worked with during the war. This finding conflicts with the conclusions in previous literature that suggests the conscription-induced interaction between male workers and female workers contributed to institutions becoming more gender balanced.


\textbf{Literature Review} \ \ Goldin and Olivetti (2013) illustrate how the impact of World War II on female labor participation varied across occupational populations, with sectors employing more highly educated workers experiencing greater persistence, while those with lower-educated workers returned to their pre-war gender ratios post-war. However, existing literature, including Acemoglu et al. (2004), does not document organizational responses to drafting due to a lack of detailed personnel information that includes lists of draftees.

Our study addresses this gap by analyzing organizational reactions at the employee level with detailed information on each employee’s position. The most closely related work is by Aneja, Farina, and Xu (2024), who analyze department-level data, to test inter-generational spillovers of leniency towards women. In contrast, this paper examines the effect on office compositions with high granularity regarding positions and occupations. We also leverage exogenous variation in exposure to female workers through a drafting mechanism, whereas Aneja et al 2024 relies on the percentage of female workers in the total workforce, a measure that is endogenous due to factors such as managerial leniency towards women or industrial/occupational attributes.

Additionally, our paper contributes to the literature on organizational responses to adverse shocks. Jaeger and Heining (2022) find that firms react to the permanent exit of an employee by altering their compensation structures based on the exiting employees' occupations. However, their study does not observe office-level restructuring, which could be a primary response to an employee's exit, as the removed worker's colleagues are already familiar with the work and can serve as immediate replacements.

Lastly, our research adds to the understanding of peer effects and career trajectories. Minni (2023) documents that working with a highly productive manager significantly enhances career outcomes, and Cullen and Perez-Truglia (2022) find that social interactions with managers increase promotion chances.

Section 2 provides background on the Tokyo civil service and their personnel policies, as well as the drafting rules during World War II. Section 3 describes the data sources and the digitization process. Section 4 discusses the empirical model, Section 5 reports the results, and Section 6 concludes.
