This paper examines the long-term effects of gender diversification in the workplace, leveraging a unique historical shock: the mass conscription of male workers during World War II in Tokyo. Using a novel, detailed personnel dataset, we explore how the forced entry of women into public service roles during wartime influenced gender dynamics and organizational structures in the years that followed.

Our findings reveal three key insights. First, wartime drafting disrupted existing workplace gender compositions, compelling organizations to rely on female employees. This resulted in a measurable and persistent increase in female representation within affected offices, even after the war ended. Second, our instrumental variable estimates demonstrate that exposure to higher female shares during the war significantly influenced long-term gender diversity within organizations. These effects likely stem from changes in attitudes toward gender integration and the normalization of female participation in professional environments.

Third, the benefits of diversification were unevenly distributed. Female employees remained concentrated in lower-tier positions and faced limited opportunities for advancement. Additionally, structural reversals occurred post-war as men reclaimed their positions, highlighting the fragility of gains achieved under temporary shocks.

By demonstrating the causal relationship between exposure to female coworkers and subsequent workplace diversification, this study contributes to the literature on labor markets, gender inequality, and the economics of organizational behavior. Our results emphasize the potential for temporary shocks to drive persistent changes in workplace diversity, while also underscoring the role of institutional and societal constraints in shaping these outcomes.