\subsection{Short term impacts of drafting}
Table~\ref{tab:drafting-hiring} reports Poisson and OLS estimates of draft-induced hiring at the position level. Columns 1 and 2 show that drafting in a position increased both total new hires and new female hires: one additional draft is associated with approximately 4--6\% more expected new hires (semi-elasticities). The effect on female hires is larger in percentage terms (5.9\%) than on total hires (4.3\%), consistent with women being hired into newly created or vacated positions. Columns 4--6 document spillover effects: drafting in neighboring positions within the same kakari (across occupation) or the same ka (within occupation) also raised hiring, with coefficients around 0.8--0.9\%. Column 7 shows heterogeneity by position type: the immediate draft effect is smaller for engineer positions than for non-engineers; the interaction term is negative and significant. The OLS specification in column 3 yields an insignificant coefficient for female hires, which is expected for count data where the Poisson model is better suited; we emphasize the Poisson results.

Table [] reports regression results on the share of women among new hires in offices with drafted and undrafted offices. Impacted offices were 18\% less likely to hire a woman to fill a vacant position than nonimpacted offices. We infer that vacancies due to drafting were more likely filled by male workers rather than female workers, since incumbent's limited experience with working with females dissuaded managers to fill the position with a female employee. This indicates offices with no drafted employees included more female workers among their new hires. The ongoing war over-burdened the Tokyo public sector and expanded their payroll to provide support to the overstrained city. We infer that women were assigned to such incremental positions and did not immediately substitute the positions traditionally held by male employees. Column 2 reports the results using the share of men drafted. The treatment effect coefficient does not change between the two different measures of the impact of the design. This indicates that male and female workers were not substitutable during the war. The number of women hired was driven by the share of drafted men rather than the share of the overall employee in an office. Columns 3 and 4 report the results with only position fixed effects or only division fixed effects. Both columns report coefficient estimates that are statistically significantly different in magnitude than the coefficient with two fixed effects. This suggests that positions and divisions/offices differed in their reactions toward hiring women to fill vacancies created by the drafted. 

We provide evidence of draft-induced hiring in Table []. The affected offices increased the number of hires by roughly 20\% more than the non-impacted offices. Additionally, we provide evidence that the share of men among employees drives the rehiring, rather than the the share of drafted employees on total number of employees (column 2). The estimates between the two share measures match each other. We strengthen our hypothesis that female and male workers were complements during the war. Lastly, the increments are similar across positions and offices. Column 3 and 4 reports estimates that exclude one of the two fixed effects, respectively. This removal changes the comparison of division-position pairs, addressing the concern that certain occupations-positions complemented their workers differently. We find similar treatment effects across both specifications and conclude that different offices and positions complemented their office size similarly to each other. However, as shown in Table [], the composition of the new hires differed between offices and positions.

Table [] reports positive effects on the number of new female employees.  Although the coefficient is insignificant across all specifications with different measures of drafted shares or fixed effect structures, drafted offices hired more female workers than non-impacted offices. Impacted offices filled their vacancies with new hires, which occasionally included women, although men filled the sudden vacancy more likely than women. 

Table [] reports regression results on the number of new hires among the total number of workers in an office. The affected offices increased their share of new hires by 20\% per unit of decrease in the share of drafted workers. Column 2 reports the results using the share of men drafted. The treatment effect coefficient does not change between the two different measures of the impact of the design. 

\subsection{Long term impacts of working with female entrants}
We provide evidence that the experience of co-working during the war did not change the share of women among their future co-workers (Table []). The first stage column conveys that the military drafting created differences in share of female among new hires during the war. The F-stat of the first stage indicates that the drafting accounts for meaningful differences in workers' war-time labor force experience. However, the IV estimate does not indicate that there is a relationship between experience and gender composition in future assignment groups. The IV estimate reports a non- statistically significant negative effect on the share of women in the employees future assignments. In particular, the OLS estimate reports a positive effect, suggesting that employees who worked with more women during the war differed from others in attributes. Hence, the introduction of female workers into the Tokyo public sector did not mitigate the skewed gender composition through wartime experiences. 

We also find that the effect is partly driven by differences in retention probabilities (Table 2). Employees from impacted offices have a lower probability of retaining a job in 1944 after the merger. The OLS and IV estimates suggest a negative effect on retention. The effect could be that exposure to women caused the employee to leave the workplace by letting their partner participate in the labor market (Aneja, Farina, and Xu 2024). If the exit was correlated with leniency towards female workers, the OLS and IV estimates should be underbiased due to the truncation of treated workers who are more willing to work with women in their workplace. 
