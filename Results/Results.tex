\subsection{Assignment of women to offices post-form}
Table~\ref{tab:exposure-manager} reports OLS estimates linking post-reform female inclusion under a manager to the manager’s wartime exposure to female coworkers. Columns 1–3 model the probability that a manager includes at least one woman. A 10 percentage point increase in exposure raises this probability by 4 points using the maximum exposure measure (column 1) and by 7 points using the mean exposure measure (column 2). Column 3 restricts the sample to departmental managers; the estimate is positive but imprecise due to the smaller sample size. Columns 4–6 report the OLS results on the extensive margin. A 10 percentage point increase in exposure is associated with 0.1–0.8 additional women in the manager's office unit, depending on the measure. We cannot estimate the effect for departmental managers because of limited statistical power.

\subsection{Draft induced vacancies and managers efforts on replacement}
Table~\ref{tab:drafting-hiring} reports Poisson estimates of how managers respond to losing a worker to military drafting. Column 1 shows that an own-position draft increases new hiring by about 4\% relative to non-impacted office units in the same department. Column 2 shows a 6\% increase in new female hires. Column 3 reports the OLS estimate for female hiring; the coefficient is positive but imprecise, which is expected because OLS is ill-suited to rare-count outcomes and many office units never hired women. The interaction column shows that engineering offices were less likely to replace drafted workers, by about 2 percentage points.

Columns 4–6 report spillovers from drafting in neighboring positions within the same office unit. Drafting increases new hires in adjacent positions by about 1\%, with no detectable change in gender composition relative to non-impacted offices. Column 7 reports spillovers across occupations within the impacted office; drafting raises new hires in other occupations by about 1\%.

\subsection{The causal effect of exposure to women on the post-reform assignment of women to offices}
Table~\ref{tab:iv-control-function} compares OLS and IV estimates of how wartime exposure to female coworkers affects the post-reform assignment of women to offices. The OLS coefficient is positive, consistent with the descriptive patterns in Table~\ref{tab:exposure-manager}, and likely reflects non-random sorting of managers into offices with different gender composition. The IV estimate, which uses drafting-induced shocks to female exposure, points in the same direction but is statistically imprecise. A key limitation is sample selection: only a small subset of managers both experienced a drafted coworker and worked alongside women, which sharply narrows the identifying variation. The first-stage relationship between drafting and female exposure is therefore weaker in the post-reform sample, reducing power and inflating standard errors.

Columns 3--4 restrict the sample to managers with wartime exposure and yield similar point estimates but even weaker first-stage strength. This underscores that selection into the overlap group is nontrivial: few managers satisfy both exposure conditions, so estimates hinge on a thin slice of the data. Taken together, the results do not allow us to reject a zero causal effect, and any interpretation must emphasize limited statistical power rather than a precise null.

\subsection{Long term impacts of working with female entrants}
