% Responses to organizational reforms vary across workplaces.
Organizational responses to policies promoting women vary across workplaces. These responses come not only from leaders but also from employees at different ranks. Men’s prior experience working with women shapes how they accommodate female hires and how much influence they exert over hiring and promotion. As a result, the existing hierarchy and attitudes inside an office largely determine whether diversification succeeds.

% Research Question:Did the unintended exposure of incumbent workers to women during the Second World War facilitate the integration of females into the tokyo civil service?
We study the rapid integration of women into the Tokyo civil service by asking whether postwar entry was concentrated in offices where male incumbents had greater wartime exposure to female coworkers. During World War II, the civil service hired many women to fill posts vacated by men who left for wartime industry or were drafted, while managerial positions remained closed because women were barred from the national exam for elite civil servants. After the war, the ban was lifted and policy expanded women’s employment, but diversification and the retention of women in managerial roles varied widely across departments, positions, and workplaces. 

% Approach: We document the post-war diversification of the Tokyo Civil service through office level responses to accomodating female workers following a removal of hiring restriction on female, while controlling for the incumbent males war-time experience with working with women.
We study how men’s wartime exposure to female coworkers shaped postwar integration of women after hiring restrictions were lifted, and whether these effects persisted in diversifying the Tokyo civil service. We assemble a new personnel record covering the universe of public servants over 50 years, with prewar, wartime, and postwar assignments, occupations, wages, and other biographical details. These data let us measure each man’s wartime coworker exposure to women and link it to his subsequent assignments and the gender composition of those workplaces. For identification, we exploit exposure induced by conscription shocks: when men were drafted, managers rapidly reshuffled staff and hired replacements, creating plausibly exogenous variation in contact with women. We then test whether men with greater exposure—especially those with more hiring discretion—later facilitated the integration and retention of women, and whether these women helped open pathways for future cohorts.


% Result: We find three set of new results. First, offices with larger exposure to women during the war accomodated women into their workplace following the lift of the hiring restriction. Second, we find that the [was it driven by contact with women?]. Three, [how did the effect persist?]
We provide three sets of results. First, offices that hired more women were more likely to be led by managers who had greater wartime exposure to female coworkers. A 10 percentage point increase in a manager’s wartime exposure is associated with nearly a 7 percent higher likelihood of working with any women in the postwar period, corresponding to about 0.2 additional women. Second, exposure to women causally increased the likelihood that wartime veterans worked with women after the war. A 10 percentage point increase in a veteran’s wartime exposure raised the number of women he later worked with by about 0.5, with larger effects for veterans who held non-managerial positions in the postwar period. Third, we find [long-term impacts].



\noindent\rule{\textwidth}{0.4pt}

% Contribution:
To examine office-level adjustments in workforce composition, we construct a novel panel of personnel records on public servants in Tokyo spanning four decades.\footnote{The data spans over the pre-war period (from 1928) to post periods (1958).} The dataset contains rich biographical information on employees, including their occupations, positions, and wartime drafting experiences.
\vspace{1em}
\color{gray}
We interpret our findings as evidence that the conscription of male workers during the war changed the gender composition of affected workplaces, primarily through the relocation of existing staff around the vacated positions. Since conscription targeted workers regardless of their rank or status in the civil service, we infer that when a key individual was drafted, their position was typically filled either by male hires with relevant experience or by transfers from nearby offices. Nevertheless, conscription generated a statistically significant shift in gender composition across employees.

Our second result suggests that the disruption caused by conscription led the coworkers of drafted employees to work in more gender-balanced offices later in their careers. Several mechanisms could explain this pattern. Greater exposure to women during the war may have reduced bias and increased men’s willingness to work alongside female colleagues—particularly in roles with hiring discretion. Alternatively, affected workers may have been reassigned to offices with higher female representation because they were perceived as more accommodating toward women.
\color{black}
\vspace{2em}

Our findings contribute to the economic history literature examining the effects of the World Wars on gender balance in the postwar economy. Recent work by \citet{NBERw32639} and \citet{doi:10.1086/719277} documents how the wartime experiences of male employees continued to shape their labor market outcomes after the war. Our paper is the first to provide evidence of such persistence in wartime experiences within an organization. Previous research, including \citet{doi:10.1086/383100}, \citet{101221}, and \citet{10.1093/restud/rdv010}, shows that regions with higher male casualties during the Second World War subsequently absorbed more women into their labor markets.\footnote{Related literature finds evidence of persisting shifts in labor market composition for blacks (\cite{doi:10.1086/716921}).} We find that this persistence also holds when the analysis is decomposed to the smallest peer unit—the office level. However, our current analysis cannot fully disentangle the mechanisms behind this persistence: workplace diversification may have resulted either from the preferences and behaviors of affected workers or from management decisions to reassign them to more gender-diverse environments.

Our paper also contributes to the literature estimating the replacement costs of workers following sudden employee exits (\cite{JaegerDeath}). A large body of research examines the costs employers face in replacing lost labor (\cite{doi:10.1086/253212}, \cite{doi:10.1086/258715}, \cite{ALAN2011973}). We extend this work by emphasizing how replacement costs depend on the internal composition of the workplace—particularly the distribution of positions across employees relative to exiting worker. Moreover, unlike traditional analyses based on private firms, we study a public sector organization characterized by rigid promotion structures and capped wages.


Our paper also contributes to the literature on how peers shape workers’ future career trajectories. Recent studies such as \cite{Minni_2025_invisiblehand} and \cite{10.1257/aer.20210863} show that assignment to high-performing managers or peers can significantly influence subordinates’ career outcomes\footnote{\cite{https://doi.org/10.1111/ecin.13278} discuss the effect of peer effects in the context of civil servants in Japan.}. We extend this literature by providing evidence on the long-term effects of working with managers who differ exogenously in their prior experience with female colleagues and examining the lasting implications of such exposure.




Section 2 provides background on the Tokyo civil service and their personnel policies, as well as the drafting rules during World War II. Section 3 describes the data sources and the digitization process. Section 4 discusses the empirical model, Section 5 reports the results, and Section 6 concludes.
