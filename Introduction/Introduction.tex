The labor shortage caused by the second world war lead to male dominated occupations to get filled by female workers. Regions with higher number of male casualties persistently increased its female labor participation, with differences across different occupations (Acemogulu, Autor, and Lyle 2004, Goldin and Olivetti, 2013)). While these findings illustrate labor market adaptations at the regional level and hint at the diversification of workplaces, the detailed processes of organizational adjustments and the long-term impacts of sudden gender diversity on future career trajectories remain under-explored.

This paper explores the organizational adjustments of a public institution during a war era, using detailed personnel records. By analyzing a unique dataset of public servants in Tokyo spanning over 30 years, we identify varying entry patterns across offices and occupations, influenced by prewar characteristics such as office size, members' job experience, and both physical and relational proximity to employees in neighboring offices. We examine how the organizational structure was modified, including the expansion of office units and roles. Additionally, we analyze the distribution of female workers across different office structures during the prewar, wartime, and postwar periods, shedding light on the evolution of gender diversity within the institution.

We then present causal evidence on how working in a gender-diverse office influences career trajectories. The personnel records include a list of employees drafted during the war, complete with their positional and occupational details. Given that the drafting process was unpredictable, it created an exogenous variation in office sizes and, consequently, gender compositions across different units. We leverage this variation to examine the impact of gender diversity in workplace experiences during the war on subsequent career outcomes, including job retention, gender diversity in future office assignments, and promotional opportunities.

We develop an algorithm that integrates multiple neural networks and computer vision technologies to detect complex layout structures in archival records and extract categorized text data. Given that most modern optical character recognition (OCR) systems are primarily trained on the English language, we utilize a neural network-based OCR developed by the National Diet Library in Japan. This OCR is adept at identifying texts within a document page, which we further analyze using named entity recognition technology to categorize information based on content and layout. We apply this algorithm to a series of historical documents to augment the biographical details of employees. This process generates a comprehensive panel data set of public sector employees in historic Tokyo, detailing their internal assignments by occupation and rank, along with biographic details for a subset of these employees.

We document a significant transformation in the organizational structure, marked by an increase in the number of departmental divisions, coinciding with a surge in female workforce participation. The number of female workers increased more than tenfold during the war but reverted to its pre-war level after the conflict ended. Notably, we observed that female employees typically entered low-level, less technically demanding positions within departments that had been predominantly male before the war.

We observe evidence that military drafting contributed to imbalances in gender composition across offices. Specifically, the removal of male workers led to a decrease in female representation in affected offices compared to those that were unaffected. This shift altered the gender dynamics among workers in both impacted and unimpacted offices. Notably, workers from offices that previously had higher female representation tended to remain in or move to offices with a female presence that was 1 percentage point greater than in offices with less initial female representation. This persistent difference highlights the long-term impact of drafting on diversity within the institution, significantly influenced by the co-working experiences of its employees.

The identification strategy relies on the randomness of the drafting process, conditional on observable controls. We provide a historical background of the drafting procedure and present balanced test results that compare the drafted and non-drafted populations. The exogenous removal of employees from offices led to changes in office composition, thereby altering the working environment for remaining employees. This removal was orthogonal to the employees’ unobserved attributes, strengthening our identification strategy. We attribute observed differences in future outcomes among employees to these draft-driven changes in co-working experiences, interpreting these as causal effects on office composition. We argue that our findings have implications for modern institutions with skewed gender distributions, suggesting that similar mechanisms could influence diversity outcomes in contemporary settings.

\

\textbf{Literature Review} \ \ Goldin and Olivetti 2013 shows the impact of the world war on female labor participation differed across different occupation population, with more higher educated sectors experiencing a higher persistence, while sectors with lower educated workers returned to their previous gender ratio after the war. However, none of the papers in the literature (Acemogulu et al 2004) documents the organizational reaction to the drafting due to the lack of rich personnel information with draftee lists. 

We contribute by analyzing the organizational reactions at the employee level with detailed information on the employees' position. The closest paper is Aneja, Farina, and Xu 2024, who analyzes department level information. Our data provides more granular information of position and occupation. We also exploit exogenous variation on exposure to female worker through a drafting mechanism. Previous literature often uses the percentage of female workers among the total workforce, which is endogenous for multiple reasons such as leniency of the manager to women or industrial/occupational attributes.

Our paper also contributes to the literature on organizational reactions to adverse shocks. Jaeger and Heining 2022 find that firms react to an unexpected permanent exit of their employee by changing their compensation structure based on the exiting employees' occupation. However, since the paper cannot observe office level restructuring, which could be the first order reaction to an workers exit since the removed workers' co-workers already knows about the content of his work and functions well as a immediate replacement. 

Lastly, our paper contributes to the literature about peer effects and career trajectories. Minni 2023 documents that working with a high productive manager vastly increases the career outcome, and Cullen and Perez-Truglia 2022 finds social interaction with managers increases the chances of promotion. 

Section 2 shares the background about the Tokyo civil service and their personnel policies, as well as the drafting rule during the second world war. Section 3 discusses the data sources and processes undertaken for digitization. Section 4 discusses the empirical model. Section 5 reports the results. Section 6 concludes.
