Despite improvements in gender equality within labor market conditions, significant disparities persist in managerial positions, with greater inequality at higher echelons. Less than 31 percent of board members in Fortune 500 firms are women, and less than 11 percent hold CEO positions\footnote{https://www.pewresearch.org/social-trends/fact-sheet/the-data-on-women-leaders/}. Research indicates that these disparities may stem from managerial biases against integrating women into the workforce (Kline, Rose, and Walters, 2022), or from negative attitudes toward female coworkers that discourage their full participation (Folke and Rickne, 2022). Additionally, traditional divisions of household chores often lead women to leave their careers prematurely, further impeding their progression to managerial roles (Goldin, 2006). This confluence of organizational biases and societal norms underscores the complex, multifaceted barriers women face in ascending to the highest echelons of corporate leadership.

Historical shocks can significantly alter the relative position of women in a community. Such shocks often displace men from the workforce and increase the presence of women. For example, the labor shortages caused by the Second World War led to women filling male-dominated occupations. Regions with a higher number of male casualties saw persistent increases in female labor participation, with variations across different occupations (Acemoglu, Autor, and Lyle, 2004; Goldin and Olivetti, 2013). Similarly, the transatlantic slave trade in Africa led to an increased presence of women in regions that had higher numbers of traded males (Manning, 1990; Teso, 2016). These examples illustrate how labor markets adapt at the regional level and hint at the diversification of workplaces, which continues to shape the current labor market landscape.

An increase in the presence of women prompts incumbent males to revise their perceptions about working with women while also reducing barriers to female workforce entry. To differentiate these two forces and quantify their impact on gender equality improvement, researchers need access to rich historical data that provides detailed information about participants' exposure to women. However, collecting such data is challenging due to poorly organized historical databases and the financial constraints associated with manually digitizing detailed personnel records.

\

This paper investigates how a sudden increase in female presence in the public sector in Tokyo has influenced gender inequality in the workplace. Leveraging recent advancements in neural-network-based OCR and computer vision technology (Dell 2024), we have constructed a novel personnel panel dataset that includes detailed compositional information about office staff. This dataset enables us to determine the gender composition of each office along with occupation, rank, and wage details for its members. Additionally, we have linked this directory to biographical records of employees, incorporating key variables such as age, educational attainment, and tenure.

We first provide descriptive evidence on the adaptation process of the Tokyo public sector. By analyzing the dataset of public servants in Tokyo spanning over 3 decades which spans across the pre-war period to the post war period, we identify varying entry patterns across offices and occupations, influenced by prewar characteristics such as office size, members' job experience, and both physical and relational proximity to employees in neighboring offices. We examine how the organizational structure was modified, including the expansion of office units and roles. Additionally, we analyze the distribution of female workers across different office structures during the prewar, wartime, and postwar periods, shedding light on the evolution of gender diversity within the institution.

\

We observe evidence that military drafting contributed to imbalances in gender composition across offices. Specifically, the removal of male workers led to a decrease in female representation in affected offices compared to those that were unaffected. This shift altered the gender dynamics among workers in both impacted and unimpacted offices. Notably, workers from offices that previously had higher female representation tended to remain in or move to offices with a female presence that was 1 percentage point greater than in offices with less initial female representation. This persistent difference highlights the long-term impact of drafting on diversity within the institution, significantly influenced by the co-working experiences of its employees.

We document a significant transformation in the organizational structure, characterized by an increase in the number of departmental divisions that coincided with a surge in female workforce participation. The number of female workers increased more than tenfold during the war, only to revert to its pre-war level after the conflict ended. Notably, our observations indicate that female employees typically entered low-level, less technically demanding positions within departments that were predominantly male before the war.

We observe evidence of increased diversification in offices where employees worked alongside a greater number of women during the war. However, our first-stage regression results indicate that the drafting of employees temporarily impeded this diversification. Offices affected by drafts saw a 0.08 percentage point decrease in female representation compared to those unaffected. This reduction was primarily due to a reorganization of office composition, where male workers from nearby offices were relocated to replace drafted employees. Remarkably, the impact on gender composition persisted beyond the war; offices that had experienced drafts exhibited a female share that was 1 percentage point higher than those that had not. This enduring difference highlights the long-term impact on diversity within the institution, significantly shaped by the co-working experiences of its employees.


\

\textbf{Literature Review} \ \ Goldin and Olivetti (2013) illustrate how the impact of World War II on female labor participation varied across occupational populations, with sectors employing more highly educated workers experiencing greater persistence, while those with lower-educated workers returned to their pre-war gender ratios post-war. However, existing literature, including Acemoglu et al. (2004), does not document organizational responses to drafting due to a lack of detailed personnel information that includes lists of draftees.

Our study addresses this gap by analyzing organizational reactions at the employee level with detailed information on each employee’s position. The most closely related work is by Aneja, Farina, and Xu (2024), who analyze department-level data. In contrast, our data offers more granular insights into positions and occupations. We also leverage exogenous variation in exposure to female workers through a drafting mechanism. Previous literature often relies on the percentage of female workers in the total workforce, a measure that is endogenous due to factors such as managerial leniency towards women or industrial/occupational attributes.

Additionally, our paper contributes to the literature on organizational responses to adverse shocks. Jaeger and Heining (2022) find that firms react to the permanent exit of an employee by altering their compensation structures based on the exiting employees' occupations. However, their study does not observe office-level restructuring, which could be a primary response to an employee's exit, as the removed worker's colleagues are already familiar with the work and can serve as immediate replacements.

Lastly, our research adds to the understanding of peer effects and career trajectories. Minni (2023) documents that working with a highly productive manager significantly enhances career outcomes, and Cullen and Perez-Truglia (2022) find that social interactions with managers increase promotion chances.

Section 2 provides background on the Tokyo civil service and their personnel policies, as well as the drafting rules during World War II. Section 3 describes the data sources and the digitization process. Section 4 discusses the empirical model, Section 5 reports the results, and Section 6 concludes.
