Despite improvements in gender equality within labor market conditions, significant disparities persist in managerial positions, with greater inequality at higher echelons. Less than 31 percent of board members in Fortune 500 firms are women, and less than 11 percent hold CEO positions\footnote{https://www.pewresearch.org/social-trends/fact-sheet/the-data-on-women-leaders/}. Research indicates that these disparities may stem from managerial biases against integrating women into the workforce (Kline, Rose, and Walters, 2022), or from negative attitudes toward female coworkers that discourage their full participation (Folke and Rickne, 2022). Additionally, traditional divisions of household chores often lead women to leave their careers prematurely, further impeding their progression to managerial roles (Goldin, 2006). This confluence of organizational biases and societal norms underscores the complex, multifaceted barriers women face in ascending to the highest echelons of corporate leadership.

Historical shocks can significantly alter the relative position of women in a community. Such shocks often displace men from the workforce and increase the presence of women. For example, the labor shortages caused by the Second World War led to women filling male-dominated occupations. Regions with a higher number of male casualties saw persistent increases in female labor participation, with variations across different occupations (Acemoglu, Autor, and Lyle, 2004; Goldin and Olivetti, 2013). Similarly, the transatlantic slave trade in Africa led to an increased presence of women in regions that had higher numbers of traded males (Manning, 1990; Teso, 2016). These examples illustrate how labor markets adapt at the regional level and hint at the diversification of workplaces, which continues to shape the current labor market landscape. An increase in the presence of women prompts incumbent males to revise their perceptions about working with women while also reducing barriers to female workforce entry. To differentiate these two forces and quantify their impact on gender equality improvement, researchers need access to rich historical data that provides detailed information about participants' exposure to women. However, collecting such data is challenging due to poorly organized historical databases and the financial constraints associated with manually digitizing detailed personnel records.

\

This paper examines the long-term impacts of a sudden increase in female workforce participation on gender inequality in Tokyo’s public sector during and after World War II. Leveraging a novel personnel dataset spanning 35 years, we analyze how wartime disruptions to office composition influenced organizational gender dynamics. Our dataset, constructed using advancements in neural-network-based optical character recognition (OCR), named entity recognition, and computer vision technology (Dell, 2024), provides unprecedented insights into the career trajectories of Tokyo’s public servants. By capturing information on occupational roles, gender composition, and hierarchical structures, the data enables us to trace the evolution of workplace diversity across nearly 800 offices. We achieve this by matching digitized employee directories through individuals' names, allowing us to track employees' office assignments and movements over time, providing a unique longitudinal view of organizational dynamics.

%Additionally, incumbent managers may relocate the incumbent male workers to different positions or roles to manage the increasing load of tasks for supporting the federal government with the on-going war. A managers' prior experience with working with female workers may have influenced the demographics of the new hire; he could prefer to hire male workers albeit the labor shortage of males in the general labor market accruing from the war. 
%We study whether the employees in office with more women presence had a more balanced pool of coworkers throughout their future career. The presence of women in their daily work environment could adjust their biases against women as their colleagues and alter their behaviors towards working with women in future offices. Workers with more co-working experience with women may choose to accept positions with more women in the working environments, fill vacant positions in their office while considering attributes of the applicants not related to gender, or promote more women to hire positions. The experience of working with a more gender neutral manager may spillover to future generations.

Using this dataset, we investigate how the conscription of male workers during World War II created vacancies that were often filled by women. These disruptions serve as a unique natural experiment to examine how exposure to female colleagues during a formative period influenced workplace diversity and career trajectories over the long term. Specifically, we analyze the initial concentration of women in lower-tier positions and how co-working experiences shaped subsequent hiring practices and promotion patterns. The analysis also considers whether these shifts persisted after the war, shedding light on the interplay between temporary shocks and lasting institutional change. To address potential endogeneity in office composition, we leverage the randomness of wartime conscription, which created unpredictable vacancies across offices and departments. By linking lists of drafted employees to the personnel dataset, we implement an instrumental variable (IV) approach to estimate the causal effect of working with women on future workplace gender balance. The use of annual variation in drafting intensities further enhances our identification strategy through an over-identified system of equations model, allowing us to disentangle the effects of wartime exposure from other confounding factors. Additionally, we test whether workers with greater exposure to women during the war were systematically assigned to offices with historically higher female representation or whether these employees themselves contributed to hiring more women in their future offices.

\

Our findings reveal that offices with higher female representation during the war demonstrated increased gender diversity in subsequent years, even after accounting for wartime reorganizations of office compositions. While drafting-induced disruptions temporarily hindered diversification by reallocating male workers from unaffected offices, offices impacted by drafting retained a 1 percentage point higher share of female employees post-war. These results highlight the potential for external shocks to drive lasting organizational changes, particularly in promoting workplace diversification.

Employees with greater exposure to female colleagues during the war were more likely to work with women in future periods. Both our OLS and IV estimates indicate positive effects, demonstrating that this outcome is not simply due to exposed workers being assigned to offices with historically higher female representation. Instead, it reflects their active role in hiring more women in their assigned offices. Furthermore, we find that workers with greater exposure to women during the war were more likely to exit the workforce compared to their less-exposed counterparts, potentially reflecting broader shifts in attitudes toward gender roles within households.

Lastly, we document a significant transformation in organizational structure during the war, marked by an increase in the number of departmental divisions that coincided with a surge in female workforce participation. The number of female workers increased more than tenfold during the war but largely reverted to pre-war levels once the conflict ended. Notably, female employees were predominantly concentrated in lower-tier, less technically demanding positions within departments that had been predominantly male prior to the war. These patterns underscore the gendered nature of workforce integration during this period and the persistence of structural barriers to female advancement.

\
\newpage
\textbf{Literature Review} \ \ Goldin and Olivetti (2013) illustrate how the impact of World War II on female labor participation varied across occupational populations, with sectors employing more highly educated workers experiencing greater persistence, while those with lower-educated workers returned to their pre-war gender ratios post-war. However, existing literature, including Acemoglu et al. (2004), does not document organizational responses to drafting due to a lack of detailed personnel information that includes lists of draftees.

Our study addresses this gap by analyzing organizational reactions at the employee level with detailed information on each employee’s position. The most closely related work is by Aneja, Farina, and Xu (2024), who analyze department-level data, to test inter-generational spillovers of leniency towards women. In contrast, this paper examines the effect on office compositions with high granularity regarding positions and occupations. We also leverage exogenous variation in exposure to female workers through a drafting mechanism, whereas Aneja et al 2024 relies on the percentage of female workers in the total workforce, a measure that is endogenous due to factors such as managerial leniency towards women or industrial/occupational attributes.

Additionally, our paper contributes to the literature on organizational responses to adverse shocks. Jaeger and Heining (2022) find that firms react to the permanent exit of an employee by altering their compensation structures based on the exiting employees' occupations. However, their study does not observe office-level restructuring, which could be a primary response to an employee's exit, as the removed worker's colleagues are already familiar with the work and can serve as immediate replacements.

Lastly, our research adds to the understanding of peer effects and career trajectories. Minni (2023) documents that working with a highly productive manager significantly enhances career outcomes, and Cullen and Perez-Truglia (2022) find that social interactions with managers increase promotion chances.

Section 2 provides background on the Tokyo civil service and their personnel policies, as well as the drafting rules during World War II. Section 3 describes the data sources and the digitization process. Section 4 discusses the empirical model, Section 5 reports the results, and Section 6 concludes.
