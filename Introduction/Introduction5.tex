
The modern state of gender inequality is shaped by historical shocks, particularly large-scale conflicts that transformed labor markets. Previous studies find that regions with higher male casualties during World War II experienced greater female labor force participation that persisted after the war (\cite{doi:10.1086/383100}, \cite{101221}, \cite{Rose_2018}). As the wars created severe shortages of male labor and expanded demand in manufacturing, many countries saw a surge in female employment—though most women left the workforce once men returned. More recent work (\cite{NBERw32639}) shows that wartime collaboration with female coworkers reshaped men’s attitudes toward women and later encouraged those around them to enter the labor market. Yet little is known about how institutions adapted to the sudden integration of women—such as by relocating incumbent workers to accommodate the influx—and whether these organizational adjustments had lasting effects. These effects could have persisted either through structural changes within workplaces or through the revised attitudes of male employees who continued to shape hiring and promotion decisions in later decades. 

We examine how conscription-driven worker exits disrupted the internal organization of offices in the short run and whether the remaining employees played a role in facilitating the postwar reintegration of women into the Tokyo civil service. Before the outbreak of the Second Sino–Japanese War in 1937, the Tokyo civil service was almost entirely male—women made up less than 2\% of the workforce and were largely confined to medical departments or short-term contracts. As wartime pressures mounted and military call-ups expanded, many male employees left either for conscription or to seek better opportunities elsewhere. The resulting labor shortages compelled offices to recruit women to fill vacant positions, introducing gender integration into many workplaces for the first time. The shift proved temporary: the end of the war and the collapse of Japan’s military brought returning men and a major downsizing of the Tokyo civil service, undoing much of the wartime integration (\Cref{fig:TimeSeries}). Yet within a decade of Japan’s surrender, the civil service again began hiring women, this time as part of a broader government initiative to promote female labor force participation.

We study how workplaces responded to the disruption caused by conscription-driven employee exits, focusing on three margins of adjustment: (1) hiring women along both the extensive and intensive margins, (2) reallocation of incumbent workers from adjacent offices, and (3) the use of promotion rules to retain employees. Exploiting exogenous variation in the timing of call-ups and in the pre-war workplace assignments of drafted employees across civil service departments, we compare hiring patterns across offices with differing exposure to the draft. During the war, the military primarily recalled veterans who had completed service before the conflict, and selection procedures were not publicly disclosed—most call-ups came as a surprise to those drafted. However, because conscription disproportionately targeted younger men, office composition and conscription risk may be correlated. To address this, we use the dataset’s rich positional and occupational information to compare neighboring workplaces within the same department that had similar pre-war occupational structures. We further conduct balance tests between conscripted and non-conscripted workers to assess differences in their prior office composition and career trajectories within the civil service.

We then examine how wartime changes in workplace gender composition shaped the subsequent career trajectories of conscripted workers’ peers. Using an instrumental variable strategy that exploits exogenous variation in the share of female coworkers assigned to non-conscripted men during the war, we study how exposure to female colleagues influenced the gender composition of each worker’s future teams. The detailed positional records and timing of female entry allow us to distinguish whether affected men were later reassigned to more female-dominated workplaces or whether they themselves facilitated greater gender integration within their units. Finally, we test whether women who worked alongside these men were more likely to advance to higher positions thereafter.

To examine office-level adjustments in workforce composition, we construct a novel panel of personnel records on public servants in Tokyo spanning four decades.\footnote{The data spans over the pre-war period (from 1928) to post periods (1958).} The dataset contains rich biographical information on employees, including their occupations, positions, and wartime drafting experiences.

\vspace{0.5em}

We find two main results. First, offices replaced only part of the workforce lost to conscription: a one-person reduction was offset by just 0.64 new hires within the same period. This pattern suggests that offices sought to fill vacancies quickly, often by reallocating staff from nearby units rather than recruiting externally. Most replacements were male—only about 6 percent of new hires associated with conscription were women. Although the overall change was modest, offices affected by conscription employed a statistically higher share of women than those that were not.

Second, we find evidence of a modest long-term impact of conscription on women’s presence in the workplace. Male employees who worked alongside a higher share of female coworkers during the war had more gender diverse pool of colleagues in the years that followed. This effect persisted for roughly fifteen years after the war before gradually fading.

\vspace{1em}
\color{gray}
We interpret our findings as evidence that the conscription of male workers during the war changed the gender composition of affected workplaces, primarily through the relocation of existing staff around the vacated positions. Since conscription targeted workers regardless of their rank or status in the civil service, we infer that when a key individual was drafted, their position was typically filled either by male hires with relevant experience or by transfers from nearby offices. Nevertheless, conscription generated a statistically significant shift in gender composition across employees.

Our second result suggests that the disruption caused by conscription led the coworkers of drafted employees to work in more gender-balanced offices later in their careers. Several mechanisms could explain this pattern. Greater exposure to women during the war may have reduced bias and increased men’s willingness to work alongside female colleagues—particularly in roles with hiring discretion. Alternatively, affected workers may have been reassigned to offices with higher female representation because they were perceived as more accommodating toward women.
\color{black}
\vspace{2em}

Our findings contribute to the economic history literature examining the effects of the World Wars on gender balance in the postwar economy. Recent work by \citet{NBERw32639} and \citet{doi:10.1086/719277} documents how the wartime experiences of male employees continued to shape their labor market outcomes after the war. Our paper is the first to provide evidence of such persistence in wartime experiences within an organization. Previous research, including \citet{doi:10.1086/383100}, \citet{101221}, and \citet{10.1093/restud/rdv010}, shows that regions with higher male casualties during the Second World War subsequently absorbed more women into their labor markets.\footnote{Related literature finds evidence of persisting shifts in labor market composition for blacks (\cite{doi:10.1086/716921}).} We find that this persistence also holds when the analysis is decomposed to the smallest peer unit—the office level. However, our current analysis cannot fully disentangle the mechanisms behind this persistence: workplace diversification may have resulted either from the preferences and behaviors of affected workers or from management decisions to reassign them to more gender-diverse environments.

Our paper also contributes to the literature estimating the replacement costs of workers following sudden employee exits (\cite{JaegerDeath}). A large body of research examines the costs employers face in replacing lost labor (\cite{doi:10.1086/253212}, \cite{doi:10.1086/258715}, \cite{ALAN2011973}). We extend this work by emphasizing how replacement costs depend on the internal composition of the workplace—particularly the distribution of positions across employees relative to exiting worker. Moreover, unlike traditional analyses based on private firms, we study a public sector organization characterized by rigid promotion structures and capped wages.


Our paper also contributes to the literature on how peers shape workers’ future career trajectories. Recent studies such as \cite{Minni_2025_invisiblehand} and \cite{10.1257/aer.20210863} show that assignment to high-performing managers or peers can significantly influence subordinates’ career outcomes\footnote{\cite{https://doi.org/10.1111/ecin.13278} discuss the effect of peer effects in the context of civil servants in Japan.}. We extend this literature by providing evidence on the long-term effects of working with managers who differ exogenously in their prior experience with female colleagues and examining the lasting implications of such exposure.




Section 2 provides background on the Tokyo civil service and their personnel policies, as well as the drafting rules during World War II. Section 3 describes the data sources and the digitization process. Section 4 discusses the empirical model, Section 5 reports the results, and Section 6 concludes.
